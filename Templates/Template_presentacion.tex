\documentclass{beamer}
\usepackage[utf8]{inputenc}

\usetheme{AnnArbor}
\usecolortheme{spruce}

\setbeamertemplate{itemize items}[default]
\setbeamertemplate{enumerate items}[default]

% define a darker brown
\definecolor{uwbrown}{HTML}{662200}
% apply dark brown to the item bullet points
\setbeamercolor{item}{fg=uwbrown}
%\usecolortheme[named=uwbrown]{structure}

\title[]{Transferencia de Calor}
\subtitle{ Ingeniería Metal Mecánica}
\author{Rubén Velázquez Hernández}
\institute[UTEQ]{Universidad Tecnológica de Querétaro}
\date{Cuatrimestre Septiembre - Diciembre 2024}
\logo{\includegraphics[width=2cm]{Imagenes/Logo_uteq}}
% \logo{\includegraphics[height=1.5cm]{/home/jordan/Pictures/wyo_logo_small.png}}

\begin{document}
 
\frame{\titlepage}

% add table of contents if wanted
%\begin{frame}
%\frametitle{Table of Contents}
%\tableofcontents
%\end{frame}
\begin{frame}
	
	\frametitle{Unidades Temáticas}
	\begin{itemize}
		\item I. Conceptos básicos de transferencia de calor
		\item II. Transferencia de calor por conducción
		\item III. Transferencia de calor por convección
		\item IV. Transferencia de calor por radiación 
		\item V. Fuentes de energía
	\end{itemize}
\end{frame}

\begin{frame}
	
\frametitle{Políticas de Clase}

\begin{enumerate}
	\item Es responsabilidad del Director de División Académica verificar, con la asistencia del Jefe de Unidad de Coordinación Académica/ Líder de Proyecto de Coordinación Académica, que se cumpla y mantenga este procedimiento.
	\item Para tener derecho a asistir a clases de las asignaturas programadas en el cuatrimestre y a la evaluación del aprendizaje correspondiente, será requisito que los alumnos estén inscritos oficialmente.
	\item Los alumnos contarán con correo institucional, que será el medio oficial para la comunicación y entrega de reportes, trabajos o actividades asignadas en la plataforma de Google.

\end{enumerate}

\end{frame}

\begin{frame}
	
	\frametitle{Políticas de Clase}
	
	\begin{enumerate}
	
		\item  El plagio de tareas, proyectos, presentaciones de equipo, evaluaciones o prácticas, queda estrictamente prohibido, el alumno que sea sorprendido entregando resultados que no sean de su autoría, perderá derecho a aprobar la evaluación correspondiente.
		\item El alumno tendrá derecho a la evaluación del aprendizaje siempre y cuando, cumpla con las actividades de enseñanza aprendizaje encomendadas y entregue en tiempo y forma los productos de aprendizaje señalados en la planeación didáctica.
		\item La puntualidad y asistencia, así como las actitudes y valores son criterios para evaluar el saber ser y aprobar la unidad en la fase ordinaria. El porcentaje mínimo de asistencia será del 80\% del total de horas de la unidad. El estudiante podrá justificar alguna inasistencia solamente en caso de incapacidad por enfermedad o a solicitud de alguna autoridad educativa, familiar o empresa debido a alguna situación especial.
	\end{enumerate}
	
\end{frame}


\begin{frame}
	
	\frametitle{Políticas de Clase}
	
	\begin{enumerate}
		
   \item La evaluación de puntualidad y asistencia, así como de actitudes y valores no aplica para evaluaciones remedial, extraordinario y de ultima asignatura, sin embargo, si estos requisitos no se cumplieron en evaluación ordinaria, estas últimas evaluaciones serán solamente SA.
	\item La emisión de documentos oficiales con resultados de aprendizaje alfanuméricos será responsabilidad de la Subdirección de Servicios Escolares y se apegará a lo señalado en los Lineamientos de Operación de Programas por Competencias emitido por la Dirección General de Universidades Tecnológicas y Politécnicas.
   \item La planeación didáctica se mantendrá vigente hasta que haya cambios en el plan de estudios o en la hoja de asignatura o cuando el Coordinador Académico de la asignatura y las academias así lo decidan.
	\item El resultado de la evaluación ordinaria de la unidad de aprendizaje será SA si todos los productos, asistencia y actitudes evaluados obtienen SA. Los niveles DE y AU se obtendrán como resultado de la evaluación con otros criterios al final de la unidad, que evidencien que los resultados de aprendizaje de la unidad son excedidos o superados en contextos diferentes.

	\end{enumerate}
	
\end{frame}

\begin{frame}
\frametitle{Evaluaciones}
\begin{itemize} 
	\item Se consideran dos evaluaciones.
	\item Las evaluaciones se harán de acuerdo a las listas de cotejo y rubricas que se presentarán anterior a la evaluación
	\item Se consideraran para tener derecho a evaluación el 100 \% de Tareas, trabajos de investigación y problemarios para resolver extra clase
	\item La evaluación por escrito no tendrá modificación de fecha 
\end{itemize}
	
\end{frame}
\begin{frame}
\frametitle{Fechas de Evaluaciones}
\begin{itemize} 
	\item Evaluación I. Conceptos Básicos y Transferencia de calor por conducción. 14-18 Octubre
	\item Evaluación II. Transferencia de calor por conveccion y radiacion  . 2-6 Diciembre
  \end{itemize}
	
\end{frame}

\begin{frame}
	\frametitle{Temas Evaluaciones}
	\begin{figure}
		\centering
		\includegraphics[width=1 \linewidth]{Imagenes/Programa}
		\label{fig:Eval1} 
	\end{figure}
\end{frame}

\end{document}