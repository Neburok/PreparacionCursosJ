\documentclass[10pt,letterpaper]{article}

% Paquetes necesarios
\usepackage[utf8]{inputenc}
\usepackage[spanish]{babel}
\usepackage{graphicx}
\usepackage{xcolor}
\usepackage{enumitem}
\usepackage{geometry}
\usepackage{titlesec}
\usepackage{fancyhdr}
% Eliminado el paquete tikz ya que no lo necesitamos más
\usepackage{tcolorbox}

% Definir colores institucionales
\definecolor{uteqGreen}{RGB}{141, 198, 63} % Color verde UTEQ #8DC63F
\definecolor{uteqGray}{RGB}{102, 102, 102} % Color gris para textos

% Configuración de márgenes
\geometry{
	letterpaper,
	top=2cm,
	bottom=2cm,
	left=2cm,
	right=2cm
}

% Configuración de encabezados y pies de página
\pagestyle{fancy}
\fancyhf{}
\renewcommand{\headrulewidth}{1pt}
\renewcommand{\headrule}{\hbox to\headwidth{\color{uteqGreen}\leaders\hrule height \headrulewidth\hfill}}
\renewcommand{\footrulewidth}{1pt}
\renewcommand{\footrule}{\hbox to\headwidth{\color{uteqGreen}\leaders\hrule height \footrulewidth\hfill}}
\fancyfoot[C]{Universidad Tecnológica de Querétaro \thepage}

% Configuración de títulos
\titleformat{\section}
{\color{uteqGreen}\normalfont\Large\bfseries}
{\thesection}{1em}{}
[{\titlerule[1pt]}]

% Configuración de listas
\setlist[enumerate]{
	leftmargin=*,
	label=\arabic*.,
	itemsep=10pt,
	parsep=0pt
}

% Se eliminaron los comandos TikZ para los logos
% En su lugar, usaremos directamente \includegraphics

% Inicio del documento
\begin{document}
	
	% Encabezado con logos lado a lado usando archivos PNG
	\begin{center}
		\begin{minipage}{0.45\textwidth}
			\begin{center}
				% Logo UTEQ a la izquierda
				% Reemplaza "logo-uteq.png" con el nombre real de tu archivo
				\includegraphics[width=0.5\textwidth]{Imagenes/Logo_uteq.png}
			\end{center}
		\end{minipage}%
		\hfill
		\begin{minipage}{0.45\textwidth}
			\begin{center}
				% Logo de 30 años a la derecha
				% Reemplaza "logo-30anios.png" con el nombre real de tu archivo
				\includegraphics[width=0.5\textwidth]{Imagenes/Logo_uteq_30.png}
			\end{center}
		\end{minipage}
		
		\vspace{0.5cm}
		{\color{uteqGreen}\huge\bfseries POLÍTICAS INSTITUCIONALES}
		
		\vspace{0.3cm}
		{\large\bfseries Universidad Tecnológica de Querétaro}
		
		\rule{\textwidth}{3pt}
	\end{center}
	
	% \vspace{0.3cm}
	
	% Sección de políticas
	\section*{\color{uteqGreen}POLÍTICAS DE CLASE}
	
	\begin{enumerate}
		\item Es responsabilidad del Director de División Académica verificar, con la asistencia del Jefe de Unidad de Coordinación Académica/ Líder de Proyecto de Coordinación Académica, que se cumpla y mantenga este procedimiento.
		
		\item Para tener derecho a asistir a clases de las asignaturas programadas en el cuatrimestre y a la evaluación del aprendizaje correspondiente, será requisito que los alumnos estén inscritos oficialmente.
		
		\item Los alumnos contarán con correo institucional, que será el medio oficial para la comunicación y entrega de reportes, trabajos o actividades asignadas en la plataforma de Google.
		
		\item La planeación didáctica debe incluir las actitudes y valores que los alumnos deberán mostrar con la finalidad de lograr una convivencia armónica en el grupo escolar y alcanzar los objetivos académicos planeados. Esta información puede ser tomada del contenido temático y complementada por la academia.
		
		\item El plagio de tareas, proyectos, presentaciones de equipo, evaluaciones o prácticas, queda estrictamente prohibido, el alumno que sea sorprendido entregando resultados que no sean de su autoría, perderá derecho a aprobar la evaluación correspondiente.
		
		\item El alumno tendrá derecho a la evaluación del aprendizaje siempre y cuando, cumpla con las actividades de enseñanza aprendizaje encomendadas y entregue en tiempo y forma los productos de aprendizaje señalados en la planeación didáctica.
		
		\item La puntualidad y asistencia, así como las actitudes y valores son criterios para evaluar el saber ser y aprobar la unidad en la fase ordinaria. El porcentaje mínimo de asistencia será del 80\% del total de horas de la unidad. El estudiante podrá justificar alguna inasistencia solamente en caso de incapacidad por enfermedad o a solicitud de alguna autoridad educativa, familiar o empresa debido a alguna situación especial.
		
		\item La evaluación de puntualidad y asistencia, así como de actitudes y valores no aplica para evaluaciones remedial, extraordinario y de ultima asignatura, sin embargo, si estos requisitos no se cumplieron en evaluación ordinaria, estas últimas evaluaciones serán solamente SA.
		
		\item La emisión de documentos oficiales con resultados de aprendizaje alfanuméricos será responsabilidad de la Subdirección de Servicios Escolares y se apegará a lo señalado en los Lineamientos de Operación de Programas por Competencias emitido por la Dirección General de Universidades Tecnológicas y Politécnicas.
		
		\item La planeación didáctica se mantendrá vigente hasta que haya cambios en el plan de estudios o en la hoja de asignatura o cuando el Coordinador Académico de la asignatura y las academias así lo decidan.
		
		\item El resultado de la evaluación ordinaria de la unidad de aprendizaje será SA si todos los productos, asistencia y actitudes evaluados obtienen SA. Los niveles DE y AU se obtendrán como resultado de la evaluación con otros criterios al final de la unidad, que evidencien que los resultados de aprendizaje de la unidad son excedidos o superados en contextos diferentes.
	\end{enumerate}
	
	\vspace{1.5cm}
	
	% Removido el logo de aniversario de aquí ya que ahora está en el encabezado
	% \begin{center}
		% \uteqaniversario
		% \end{center}
	
	\vspace{1.5cm}
	
	% Pie de página adicional
	\begin{center}
		{\small Liga: GA-FM}
	\end{center}
	
\end{document}